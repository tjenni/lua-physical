%!TEX program = lualatex
\documentclass{ltxdoc}

\usepackage{url}
\usepackage[english]{babel}
\usepackage{hyperref}
\usepackage{luacode}
\usepackage{listings}
\usepackage{siunitx}
\usepackage{tabularx}
\usepackage{float}
\usepackage{ulem}
\usepackage{xcolor}
\usepackage{framed}
\usepackage{lscape}
\usepackage{ifthen}


\begin{luacode*}
physical = require("physical")
N = physical.Number
\end{luacode*}

\newcommand{\q}[1]{%
  \directlua{tex.print(physical.Quantity.tosiunitx(#1,"add-decimal-zero=true,scientific-notation=fixed,exponent-to-prefix=false"))}%
}

\newcommand{\qs}[1]{%
  \directlua{tex.print(physical.Quantity.tosiunitx(#1,"scientific-notation=true,exponent-to-prefix=false,round-integer-to-decimal=true"))}%
}

\newcommand{\qu}[1]{%
  \directlua{tex.print(physical.Quantity.tosiunitx(#1,nil,2))}%
}

\DeclareSIUnit\number{}


% siunitx config
\sisetup{
	output-decimal-marker = {.}, 
	per-mode = symbol,
	separate-uncertainty = true,
	add-decimal-zero = true,
	exponent-product = \cdot,
	round-mode=off
}

\lstdefinelanguage{lua}
{
  morekeywords={
    for,end,function,do,if,else,elseif,then,
    tex.print,tex.sprint,io.read,io.open,string.find,string.explode,require
  },
  morecomment=[l]{--},
  morecomment=[s]{--[[}{]]},
  morestring=[b]''
}

\lstset{
  numberstyle=\footnotesize\color{gray},
  keywordstyle=\ttfamily\bfseries\color{blue},
  basicstyle=\ttfamily\footnotesize,
  commentstyle=\itshape\color{gray},
  stringstyle=\ttfamily,
  tabsize=2,
  numbers=left,
  showstringspaces=false,
  breaklines=true,
  framexleftmargin=20pt,
  framexrightmargin=3.5pt,
  xleftmargin=24.6pt,
  breakindent=30pt,
  morekeywords={},
  frame=leftline,
  framerule=2pt,
  rulecolor=\color{blue},
  mathescape=false,
  captionpos=t,
  escapechar=`
}



% no paragraph indent
\setlength\parindent{0pt}

% set emph italic
\renewcommand{\emph}[1]{\textit{#1}}

% lualatex logo
\newcommand{\LuaLaTeX}{Lua\LaTeX}

% left bar
\renewenvironment{leftbar}[1][\hsize]
{%
    \def\FrameCommand
    {%
        {\color{blue!40!white}\vrule width 2pt}%
        \hspace{10pt}%must no space.
        \fboxsep=\FrameSep%\colorbox{yellow}
    }%
    \MakeFramed{\hsize#1\advance\hsize-\width\FrameRestore}%
}
{\endMakeFramed}

% style for table header
\newcommand\thead[1]{#1}


\begin{document}

  \lstset{language=[LaTex]Tex}


	\title{The \textsc{lua-physical} library \\\ \\\normalsize Version 0.1}
	\author{Thomas Jenni}
	\date{\today}
	\maketitle



\begin{abstract}
\noindent |lua-physical| is a pure Lua library which provides functions and object for doing computation with physical quantities. It has been written to simplify the creation problem sets. The package provides units of the SI and the imperial system. Furthermore an almost complete set of international currencies are supported, however without online exchange rates. In order to display the numbers with measurement uncertainties, the package is able to perform gaussian error propagation.
\end{abstract}

\tableofcontents






\newpage
\section{Introduction}

The author of this package is a teacher at the high school \emph{Kantonsschule Zug} in Switzerland. The main use of this package is to write physics problem sets. \LuaLaTeX does make it possible to integrate physical calculations directly. The package has been in use since 2016. Many bugs have been found and fixed. Nevertheless it still is possible, that some were not found yet. Therefore the author recommends not to use this package industry or science. If one does so, it's the responsability of the user to check results for plausability. If the user finds some bugs, they can be reported at github.com or directly to the author (\url{thomas.jenni (at) ksz.ch}).



\section{Loading}

This package is a pure Lua library. Therefore one has to require it explicitly by calling |require("physical")|. For printing results, the |siunitx| package can be used. It's recommended to define a macro like |\q| to convert the lua quantity object to a |siunitx| expression.

The following Latex preambel loads the |lua-physical| package and creates a macro |\q| for printing physical quantities.
\nopagebreak
\begin{lstlisting}[language=Tex, caption=basic preamble]
  \usepackage{lua-physical}
  \usepackage{siunitx}

  % configure siunitx
  \sisetup{
    output-decimal-marker = {.}, 
    per-mode = symbol,
    separate-uncertainty = true,
    add-decimal-zero = true,
    exponent-product = \cdot,
    round-mode = off
  }

  % load lua-physical package
  \begin{luacode*}
    physical = require("physical")
    N = physical.Number
  \end{luacode*}

  % print physical quantities
  \newcommand{\q}[1]{%
    \directlua{tex.print(physical.Quantity.tosiunitx(#1,"scientific-notation=fixed,exponent-to-prefix=false"))}%
  }
\end{lstlisting}


\subsection{Dependencies}

This is a standalone package. If a pretty print to \LuaLaTeX{} is wanted, the package |siunitx| sould be installed. 


\section{Usage}

Given the basic preamble, units can be used in lua code directly. By convention, all units have an underscore in front of them, i.e. Meter is |_m|, Second is |_s|. All available units are listed in chapter \ref{ch:Supported Units}. The Result of the calculation can be printed to \LuaLaTeX{} by using the macro |\q{}|.

\begin{lstlisting}[caption=The velocity of a car.,label=lst:Car Velocity]
\begin{luacode}
  s = 10 * _m
	t = 2 * _s
	v = s/t
\end{luacode}

A car travels $\q{s}$ in $\q{t}$. calculate its velocity.
$$
	v=\frac{s}{t} = \frac{\q{s}}{\q{t}} = \q{v}
$$
\end{lstlisting}

\begin{luacode}
  s = 10 * _m
  t = 2 * _s

  v = s/t
\end{luacode}

\leftbar
A car travels $\q{s}$ in $\q{t}$. Calculate its velocity.
$$
  v=\frac{s}{t} = \frac{\q{s}}{\q{t}} = \q{v}
$$
\endleftbar

In the above listing \ref{lst:Car Velocity}, the variable |s| stands for displacement and has the unit meter |_m|. The variable |t| stands for time and is given in second |_s|. If mathematical operations are done on them, new physical quantities are created. In the problem above, the velocity |v| is calculated by dividing |s| by |t|. The instance |v| has the derived unit \si{\m\per\s}. By using the macro |\q{}| all quantities can be printed to the \LuaLaTeX{} code directly.


\subsection{Unit conversion}

Very often the result of a calculation is needed in different unit, than the given quantities are. In the following listing \ref{lst:Volume of Cuboid} the task is to calculate the volume of a cuboid with lengths given in different units. If the volume is calculated by multipling all three lengths, the unit of the result is \si{\cm\mm\m}. If the unit \si{\cm\cubed} is preferred, it has to be converted explicitly. The conversion function is called |to()| and is available on all physical quantitiy instances. At first this looks a bit cumbersome. The reason of this behaviour is, that the software is not able to guess the unit of the result. In many cases, like in the example here, it's not clear what unit the result sould have. Therefore the user has always to give the target unit explicitly.

\begin{lstlisting}[caption=The volume of a cuboid.,label=lst:Volume of Cuboid]
\begin{luacode}
  a = 12 * _cm
  b = 150 * _mm
  c = 1.5 * _m
  
  V = a*b*c
\end{luacode}

Find the volume of a rectangular cuboid with lengths $\q{a}$, 
$\q{b}$ and $\q{c}$.
$$
  V= a \cdot b \cdot c
  = \q{a} \cdot \q{b} \cdot \q{c}
  = \q{V}
  = \uuline{\q{V:to(_dm^3)}}
$$
\end{lstlisting}

\begin{luacode}
  a = 12 * _cm
  b = 150 * _mm
  c = 1.5 * _m
  
  V = a*b*c
\end{luacode}

\leftbar
Find the volume of a rectangular cuboid with lengths $\q{a}$, $\q{b}$ and $\q{c}$.
$$
  V= a \cdot b \cdot c
  = \q{a} \cdot \q{b} \cdot \q{c}
  = \q{V}
  = \uuline{\q{V:to(_dm^3)}}
$$
\endleftbar




\subsubsection{Temperature Conversion}

In the following problem, listing \ref{lst:degree to kelvin} , the task is to convert a temperature given in the unit degree Celsius to Kelvin. As can be seen in the listing, the conversion function has two parameters. 

The first argument is the target unit. The second is a boolean that tells the |to|-function to call a unit specific conversion function. By default the second argument is |false|. 

Most units do not have a conversion function. Exceptions are the unit degree Celsius |_degC| and degree Fahrenheit |_degF|. These units are ambigous and can be interpreted as temperature differences or as an absolute temperatures. In the latter case, the conversion to base units is not a linear, but an affine transformation. This is because degree Celsius and degree Fahrenheit scales have their zero points at different temperatures compared to the unit Kelvin. Therefore these units have their own conversion functions. 

By default |_degC| and |_degF| units are standing for temperature differences. If one wants to have it converted absolutely, the conversion function |to()| should have the second argument set to true.

\begin{lstlisting}[caption=Temperature conversion.,label=lst:degree to kelvin]
\begin{luacode}
  T = 20 * _degC
\end{luacode}

A thermometer shows $\q{T}$. Converte this quantity to Kelvin. 
$$
  T = \q{T:to(_K)} + \q{273.15 * _K}
  =  \q{T:to(_K,true)}
$$
\end{lstlisting}

\begin{luacode}
  T = 20 * _degC
\end{luacode}

\leftbar
A thermometer shows $\q{T}$. Converte this quantity to Kelvin. 
$$
  T = \q{T:to(_K)} + \q{273.15 * _K}
  =  \q{T:to(_K,true)}
$$
\endleftbar










\subsubsection{Uncertainty}

The package supports uncertainty propagation. To create a number with an uncertainty, an instance of |physical.Number| has to be created, see listing \ref{lst:rectangular area}. It has to be remembered, that |N| is a alias for |physical.Number|. The first argument of the constructor |N(mean, uncertainty)| is the mean value and the second one the uncertainty of the measurement. If the proposed preamble is used, the uncertainty is by default seperated from the mean value by a plus-minus sign.

For the uncertainty propagation the gaussian formula 
$$
  \Delta f = \sqrt{ \left(\frac{\partial f}{x_1} \cdot \Delta x_1\right)^2 + \dots + \left(\frac{\partial f}{x_n} \cdot \Delta x_2 \right)^2 }
$$
is used. This formula is a good estimation for the uncertainty $\Delta f$, if the quantities $x_1, \dots, x_n$ the function $f$ depends on, have no correlation. Further, the function $f$ has to behave linear, if the quantities $x_i$ are changed in the range of their uncertainties.
 


\begin{lstlisting}[caption=Uncertainty in area calculation.,label=lst:rectangular area]
\begin{luacode}
  a = N(2,0.1) * _m
  b = N(3,0.1) * _m

  A = (a*b):to(_m^2)
\end{luacode}

Calculate the area of a rectangle with lengths $\q{a}$ and $\q{b}$.
$$
  A = a \cdot b 
  = \q{a} \cdot \q{b}
  = \uuline{\q{A}}
$$
\end{lstlisting}

\begin{luacode}
  a = N(2,0.1) * _m
  b = N(3,0.1) * _m

  A = (a*b):to(_m^2)
\end{luacode}

\leftbar
Calculate the area of a rectangle with lengths $\q{a}$ and $\q{b}$.
$$
  A = a \cdot b 
  = \q{a} \cdot \q{b}
  = \uuline{\q{A}}
$$
\endleftbar


Instead of printing always the uncertainties, one can use the uncertainty calculation to provide significant numbers.

In the following problem, listing \ref{lst:volume of ideal gas} , the task is to calculate the volume of an ideal gas. Given are pressure |p| in |_bar|, amount of substance |n| in |_mol| and temperature |T| in degree celsius |_degC|. In order to do the calculation, one has to convert |T|, which is given as an absolute temperature in degree celsius to the base unit Kelvin first. By setting |N.omitUncertainty = true|, all uncertainties are not printed.

\begin{lstlisting}[caption=Volume of an ideal gas.,label=lst:volume of ideal gas]
\begin{luacode}
  N.omitUncertainty = true
  p = N(1.013,0.0001) * _bar
  n = N(1,0.01) * _mol
  T = N(30,0.1) * _degC
  
  V = ( n * _R * T:to(_K,true) / p ):to(_L)
\end{luacode}

An ideal gas ($\q{n}$) has a pressure of $\q{p}$ and a temperature of $\q{T}$. Calculate the volume of the gas.
$$
  V=\frac{ \q{n} \cdot \q{_R} \cdot \q{T:to(_K,true)} }{ \q{p} }
  = \q{V}
  = \uuline{\q{V}}
$$
\end{lstlisting}


\begin{luacode}
  N.omitUncertainty = true
  p = N(1.013,0.0001) * _bar
  n = N(1,0.01) * _mol
  T = N(30,0.1) * _degC
  
  V = ( n * _R * T:to(_K,true) / p ):to(_L)
\end{luacode}

\leftbar
An ideal gas ($\q{n}$) has a pressure of $\q{p}$ and a temperature of $\q{T}$. Calculate the volume of the gas.

$$
  V=\frac{ \q{n} \cdot \q{(_R*N(1,0.001)):to(_J/(_mol*_K))} \cdot \q{T:to(_K,true)} }{ \q{p} }
  = \uuline{\q{V}}
$$
\endleftbar





















\renewcommand{\arraystretch}{1.5}


% define macros for dimensions

\newcommand{\M}[1][]{\mathrm{M}^{#1}\,}

\renewcommand{\L}[1][]{\mathrm{L}^{#1}\,}

\newcommand{\T}[1][]{\mathrm{T}^{#1}\,}

\newcommand{\I}[1][]{\mathrm{I}^{#1}\,}

\renewcommand{\K}[1][]{\mathrm{\Theta}^{#1}\,}

\renewcommand{\N}[1][]{\mathrm{N}^{#1}\,}

\renewcommand{\J}[1][]{\mathrm{J}^{#1}\,}

\newcommand{\B}[1][]{\mathrm{B}^{#1}\,}

\newcommand{\C}[1][]{\mathrm{C}^{#1}\,}

\newcommand{\1}{\mathrm{1}}




\newpage
\section{Supported Units}
\label{ch:Supported Units}

All supported units are listed in this chapter. Subchapter \ref{ch:Base Units} lists the seven base units of the International System of Units (SI). In subchapter \ref{ch:Constants}  mathematical and physical constants are defined. The subchapter \ref{ch:Derived Units SI} contains all derived units from the SI system. Subchapter \ref{ch:Units outside SI} lists units, which are common but outside of the SI system. The subchapters \ref{ch:Imperial Units} and \ref{ch:U.S. customary units} are dedicated to imperial and U.S. customary units. The last subchapter \ref{ch:currencies} containts international currencies.

\subsection{Base Units}
\label{ch:Base Units}


\begin{table}[H]
\centering
\begin{tabularx}{\linewidth}{%
  >{\setlength\hsize{0.5\hsize}}X%
  l%
  l%
  l%
  >{\setlength\hsize{1.5\hsize}}X%
}

\thead{Quantity} & \thead{Unit} & \thead{Symbol} & \thead{Dim.} & \thead{Definition} \\\hline

time &
second & |_s| & 
$\T$ & 
The SI unit of time. It is defined by taking the fixed numerical value of the caesium frequency $\Delta \nu_{Cs}$, the unperturbed ground-state hyperfine transition frequency of the caesium 133 atom, to be \num{9192631770} when expressed in the unit $\qu{_s^-1}$. \\

length &
meter & 
|_m| & 
$\L$ &
The SI unit of length. It is defined by taking the fixed numercial value of the speed of light in vacuum $c$ to be $\num{299792458}$ when expressed in the unit of $\q{_m/_s}$. \\


mass &
kilogram  &  
|_kg| & 
$\M$ & 
The SI unit of mass. It is defined by taking the fixed numerical value of the Planck constant $h$ to be $\qs{(_h_P/(_J*_s)):to()}$ when expressed in $\qu{_m^2*_kg/_s}$.\\


electric \newline current &
ampere & 
|_A| & 
$\I$ & 
The SI unit of electric current. It is defined by taking the fixed numerical value of the elementary charge $e$ to be $\qs{(_e/_C):to()}$ when expressed in $\qu{_A*_s}$.\\


thermodynamic \newline temperature &
kelvin  & 
|_K| & 
$\K$ & 
The SI unit of the thermodynamic temperature. It is defineed by taking the fixed numerical value of the Boltzmann constant $k_B$ to be $\qs{(_k_B/(_J/_K)):to()}$ when expressed in $\q{_kg*_m^2*_s^-2*_K^-1}$\\


\hline

\end{tabularx}
\end{table}

\begin{table}[H]
\centering
\begin{tabularx}{\linewidth}{%
  >{\setlength\hsize{0.5\hsize}}X%
  l%
  l%
  l%
  >{\setlength\hsize{1.5\hsize}}X%
}

\thead{Quantity} & \thead{Unit} & \thead{Symbol} & \thead{Dim.} & \thead{Definition} \\\hline


amount of \newline substance &
mole  & 
|_mol| &  
$\N$ & 
The SI unit of amount of substance. One mole contains exactly $\qs{(_N_A*_mol):to()}$ elementary entities. This number is the fixed numerical value of the Avogadro constant $N_A$ when expressed in $\qu{1/_mol}$.\\


luminous \newline intensity &
candela & 
|_cd| & 
$\J$ & 
The SI unit of luminous intensity in a given direction. It is defined by taking the fixed numerical value of the luminous efficacy of monochromatic radiation of frequency $\qs{540e12 * _Hz}$, $K_{cd}$, to be $683$ when expressed in the unit $\qu{_cd*_sr*_kg^-1*_m^-2*_s^3}$.\\


number &
-- &
|_1| &
$\1$ & 
The dimensionless number one.  \\


information &
bit & 
|_bit| & 
$\B$ &
The smallest amount of information. \\


currency &
euro & 
|_EUR| & 
$\C$ &
The value of the currency Euro. \\\hline

\end{tabularx}
\caption{Base units of the International System of Units (SI) \cite{bipm18}, information,  currency and the dimensionless number one.}
\end{table}





\begin{landscape}
\subsection{Constants}
\label{ch:Constants}

Almost all physical constants are taken from the NIST webpage \cite{nist19}. The nominal values of solar, terrestrial and jovial quantities are taken from IAU Resolution B3 \cite{iau16}.

\begin{table}[H]
\centering
\begin{tabularx}{\linewidth}{%
  >{\setlength\hsize{0.6\hsize}}X%
  l%
  l%
  >{\setlength\hsize{1.4\hsize}}X%
  c%
}

\thead{Name} & \thead{Symbol} & \thead{Dim.} & \thead{Definition} & \thead{Source} \\\hline

|pi| &
|_Pi| &
$\1$ & 
|3.1415926535897932384626433832795028841971 * _1|  \\

|eulersnumber| &
|_E| &
$\1$ & 
|2.7182818284590452353602874713526624977572 * _1|  \\

|speedoflight| &
|_c| &
$\L[1]\T[-1]$ & 
|299792458 * _m/_s| &
\cite{nist19} \\


|gravitationalconstant| &
|_Gc| &
$\L[3]\M[-1]\T[-2]$ & 
|N(6.67408e-11,3.1e-15) * _m^3/(_kg*_s^2)| &
\cite{nist19} \\

 
|planckconstant| &
|_h_P| &
$\L[2]\M[1]\T[-1]$ & 
|6.62607015e-34 * _J*_s| &
\cite{nist19} \\


|reducedplanckconstant| &
|_h_Pbar| &
$\L[2]\M[1]\T[-1]$ &
|_h_P/(2*Pi)|  &
\cite{nist19} \\


|elementarycharge| &
|_e| &
$\T[1]\I[1]$ & 
|1.602176634e-19 * _C| &
\cite{nist19} \\


|vacuumpermeability| &
|_u_0| &
$\L[1]\M[1]\T[-2]\I[-2]$ & 
|4e-7*Pi * _N/_A^2| &
\cite{nist19} \\

|vacuumpermitivity| &
|_e_0| &
$\L[-3]\M[-1]\T[4]\I[2]$ & 
|1/(_u_0*_c^2)| &
\cite{nist19} \\

|atomicmassunit| &
|_u| &
$\M[1]$ & 
|N(1.66053904e-27, 2e-35) * _kg| &
\cite{nist19} \\

|electronmass| &
|_m_e| &
$\M[1]$ & 
|N(9.10938356e-31, 1.1e-38) * _kg| &
\cite{nist19} \\

|protonmass| &
|_m_p| &
$\M[1]$ & 
|N(1.672621898e-27, 2.1e-35) * _kg|  &
\cite{nist19} \\

|neutronmass| &
|_m_n| &
$\M[1]$ & 
|N(1.674927471e-27, 2.1e-35) * _kg| &
\cite{nist19} \\

|bohrmagneton| &
|_u_B| &
$\L[2]\I[1]$ & 
|_e*_h_Pbar/(2*_m_e)| &
\cite{nist19} \\

|nuclearmagneton| &
|_u_N| &
$\L[2]\I[1]$ & 
|_e*_h_Pbar/(2*_m_p)| &
\cite{nist19} \\


\hline

\end{tabularx}
\end{table}



\begin{table}[H]
\centering
\begin{tabularx}{\linewidth}{%
  >{\setlength\hsize{0.8\hsize}}X%
  l%
  l%
  >{\setlength\hsize{1.2\hsize}}X%
  c%
}

\thead{Name} & \thead{Symbol} & \thead{Dim.} & \thead{Definition} & \thead{Source} \\\hline


|electronmagneticmoment| &
|_u_e| &
$\L[2]\I[1]$ & 
|N(-928.4764620e-26,5.7e-32) * _J/_T|  &
\cite{nist19} \\

|protonmagneticmoment| &
|_u_p| &
$\L[2]\I[1]$ & 
|N(1.4106067873e-26,9.7e-35) * _J/_T|  &
\cite{nist19} \\

|neutronmagneticmoment| &
|_u_n| &
$\L[2]\I[1]$ & 
|N(-0.96623650e-26,2.3e-26) * _J/_T|  &
\cite{nist19} \\

|finestructureconstant| &
|_alpha| &
$\1$ & 
|_u_0*_e^2*_c/(2*_h_P)|  &
\cite{nist19} \\

|rydbergconstant| &
|_Ry| &
$\L[-1]$ & 
|_alpha^2*_m_e*_c/(2*_h_P)|  &
\cite{nist19} \\

|avogadronumber| &
|_N_A| &
$\N[-1]$ & 
|6.02214076e23/_mol|  &
\cite{nist19} \\

|boltzmannconstant| &
|_k_B| &
$\L[2]\M[1]\T[-2]\K[-1]$ & 
|1.380649e-23 * _J/_K|  &
\cite{nist19} \\

|molargasconstant| &
|_R| &
$\L[2]\M[1]\T[-2]\K[-1]\N[-1]$ & 
|N(8.3144598, 4.8e-6) * _J/(_K*_mol)|  &
\cite{nist19} \\

|stefanboltzmannconstant| &
|_sigma| &
$\L[2]\M[1]\T[-2]\K[-1]\N[-1]$ & 
|Pi^2*_k_B^4/(60*_h_Pbar^3*_c^2)|  &
\cite{nist19} \\

|standardgravity| &
|_g_0| &
$\L[1]\T[-2]$ & 
|9.80665 * _m/_s^2|  &
\cite{nist19} \\



|nomsolradius| &
|_R_S_nom| &
$\L[1]$ & 
|6.957e8 * _m|  &
\cite{iau16} \\

|nomsolirradiance| &
|_S_S_nom| &
$\M[1]\T[-3]$ & 
|1361 * _W/_m^2|  &
\cite{iau16} \\

|nomsolluminosity| &
|_L_S_nom| &
$\L[2]\M[1]\T[-3]$ & 
|3.828e26 * _W|  &
\cite{iau16} \\

|nomsolefftemperature| &
|_T_S_nom| &
$\K[1]$ & 
|5772 * _K|  &
\cite{iau16} \\

|nomsolmassparam| &
|_GM_S_nom| &
$\L[3]\T[-2]$ & 
|1.3271244e20 * _m^3 * _s^-2|  &
\cite{iau16} \\

|nomterreqradius| &
|_Re_E_nom| &
$\L[1]$ & 
|6.3781e6 * _m|  &
\cite{iau16} \\

|nomterrpolradius| &
|_Rp_E_nom| &
$\L[1]$ & 
|6.3568e6 * _m|  &
\cite{iau16} \\

|nomterrmassparam| &
|_GM_E_nom| &
$\L[3]\T[-2]$ & 
|3.986004e14 * _m^3 * _s^-2|  &
\cite{iau16} \\


\hline

\end{tabularx}
\end{table}






\begin{table}[H]
\centering
\begin{tabularx}{\linewidth}{%
  >{\setlength\hsize{0.8\hsize}}X%
  l%
  l%
  >{\setlength\hsize{1.2\hsize}}X%
  c%
}

\thead{Name} & \thead{Symbol} & \thead{Dim.} & \thead{Definition} & \thead{Source} \\\hline




|nomjovianeqradius| &
|_Re_J_nom| &
$\L[1]$ & 
|7.1492e7 * _m|  &
\cite{iau16} \\

|nomjovianpolradius| &
|_Rp_J_nom| &
$\L[1]$ & 
|6.6854e7 * _m|  &
\cite{iau16} \\

|nomjovianmassparam| &
|_GM_J_nom| &
$\L[3]\T[-2]$ & 
|1.2668653e17 * _m^3*_s^-2|  &
\cite{iau16} \\

\hline

\end{tabularx}
\caption{Physical and mathematical constants.}
\end{table}


\end{landscape}




\begin{luacode}
function getdim(q)
  local str = q.dimension:__tostring()
  
  str = string.gsub(str,"%[","")
  str = string.gsub(str,"%]","")

  return str
end
\end{luacode}


\newcommand{\printunit}[3][]{
  \ifthenelse{\equal{#1}{}}{
    \directlua{tex.print(getdim(#2))}
  }{
    #1
  } &
  \directlua{tex.print(#2.unit.name)} & 
  |#2| &
  |#3| \\
}




\newpage
\subsection{Coherent derived units in the SI}
\label{ch:Coherent derived units in the SI}

All units in this section are coherent derived units from the SI base units with special names, \cite[118]{bipm06}. 

\begin{table}[H]
\centering
\begin{tabularx}{\linewidth}{%
  >{\setlength\hsize{1.2\hsize}}X%
  l%
  l%
  >{\setlength\hsize{0.8\hsize}}X%
}

\thead{Quantity} & \thead{Unit} & \thead{Symbol} & \thead{Definition} \\\hline


\printunit[Plane Angle\protect\footnotemark]{_rad}{_1}
\printunit[Solid Angle\protect\footnotemark]{_sr}{_rad^2}
\printunit{_Hz}{1/_s}
\printunit{_N}{_kg*_m/_s^2}
\printunit{_Pa}{_N/_m^2}
\printunit[Energy]{_J}{_N*_m}
\printunit{_W}{_J/_s}
\printunit{_C}{_A*_s}
\printunit{_V}{_J/_C}
\printunit{_F}{_C/_V}
\printunit{_Ohm}{_V/_A}
\printunit[Electric Conductance\protect\footnotemark]{_S}{_A/_V}
\printunit{_Wb}{_V*_s}
\printunit{_T}{_Wb/_m^2}
\printunit{_H}{_Wb/_A}
\printunit[Celsius Temperature\protect\footnotemark]{_degC}{_K}
\printunit[Luminous Flux]{_lm}{_cd*_sr}
\printunit{_lx}{_lm/_m^2}

\hline

\end{tabularx}
\end{table}

\footnotetext[1]{
  In the SI system, the quantity Plane Angle has the dimension of a number.
}

\footnotetext[2]{
  In the SI system, the quantity Solid Angle has the dimension of a number.
}

\footnotetext[3]{
  The unit \texttt{\_PS} stands for peta siemens and is in conflict with the german version of the unit horsepower (Pferdestärke). Since the latter is more common than peta siemens, \texttt{\_PS} is defined as the german version of horsepower.
}

\footnotetext[3]{
  The unit \texttt{\_degC} is by default interpreted as a temperature difference. For absolute temperature conversion, set the second parameter of the \texttt{to}-function, i.e. \texttt{(15*\_degC):to(\_K,true)}.
}


\begin{table}[H]
\centering
\begin{tabularx}{\linewidth}{%
  >{\setlength\hsize{1.2\hsize}}X%
  l%
  l%
  >{\setlength\hsize{0.8\hsize}}X%
}

\thead{Quantity} & \thead{Unit} & \thead{Symbol} & \thead{Definition} \\\hline

\printunit[Activity]{_Bq}{1/_s}
\printunit{_Gy}{_J/_kg}
\printunit[Dose Equivalent]{_Sv}{_J/_kg}
\printunit{_kat}{_mol/_s}

\hline

\end{tabularx}
\caption{Coherent derived units of the SI}
\end{table}





% https://www.bipm.org/utils/common/pdf/si_brochure_8_en.pdf

\subsection{Non-SI units accepted for use with the SI}
\label{ch:Non-SI units accepted for use with the SI}

There are a few units with dimension $\1$. The unit Bel is only available with prefix decibel, because |_B| is the unit byte.

\begin{table}[H]
\centering
\begin{tabularx}{\linewidth}{%
  >{\setlength\hsize{1.2\hsize}}X%
  l%
  l%
  >{\setlength\hsize{0.8\hsize}}X%
}

\thead{Quantity} & \thead{Unit} & \thead{Symbol} & \thead{Definition} \\\hline

\printunit[Time]{_min}{60*_s}
\printunit[ ]{_h}{60*_min}
\printunit[ ]{_d}{24*_h}

\printunit[Plane Angle]{_deg}{(_Pi/180)*_rad}
\printunit[ ]{_arcmin}{_deg/60}
\printunit[ ]{_arcsec}{_arcmin/60}

\printunit{_hectare}{1e4*_m^2}

\printunit{_L}{1e-3*_m^3}

\printunit[Mass]{_t}{1e3*_kg}


\hline

\end{tabularx}

\caption{Units outside of the International System of Units (SI)}

\end{table}









\subsection{Other Non-SI units}
\label{ch:Other Non-SI units}

\begin{table}[H]
\centering
\begin{tabularx}{\linewidth}{%
  >{\setlength\hsize{1.2\hsize}}X%
  l%
  l%
  >{\setlength\hsize{0.8\hsize}}X%
}

\thead{Quantity} & \thead{Unit} & \thead{Symbol} & \thead{Definition} \\\hline


\printunit[Time]{_svedberg}{1e-13*_s}
\printunit[ ]{_wk}{7*_d}
\printunit[ ]{_a}{365.25*_d}

\printunit[Length]{_angstrom}{1e-10*_m}
\printunit[ ]{_fermi}{1e-15*_m}
\printunit[ ]{_au}{149597870700*_m}
\printunit[ ]{_ls}{_c*_s}
\printunit[ ]{_ly}{_c*_a}
\printunit[ ]{_pc}{(648000/_Pi)*_au}

\printunit{_barn}{1e-28*_m^2}
\printunit[ ]{_are}{1e2*_m^2}

\printunit{_tsp}{5e-3*_L}
\printunit[ ]{_Tbsp}{3*_tsp}

\printunit[Plane Angle]{_gon}{(Pi/200)*_rad}
\printunit[ ]{_tr}{2*Pi*_rad}

\printunit[Solid Angle]{_sp}{4*Pi*_sr}

\hline

\end{tabularx}
\end{table}





\begin{table}[H]
\centering
\begin{tabularx}{\linewidth}{%
  >{\setlength\hsize{1\hsize}}X%
  l%
  l%
  >{\setlength\hsize{1\hsize}}X%
}

\thead{Quantity} & \thead{Unit} & \thead{Symbol} & \thead{Definition} \\\hline

\printunit{_bar}{1e5*_Pa}
\printunit[ ]{_atm}{101325*_Pa}
\printunit[ ]{_at}{_kp/_cm^2}
\printunit[ ]{_mmHg}{133.322387415*_Pa}
\printunit[ ]{_Torr}{(101325/760)*_Pa}

\printunit{_kp}{_kg*_g_0}

\printunit[Energy]{_cal}{4.184*_J}
\printunit[ ]{_cal_IT}{4.1868*_J}
\printunit[ ]{_g_TNT}{1e3*_cal}
\printunit[ ]{_t_TNT}{1e9*_cal}
\printunit[ ]{_eV}{_e*_V}
\printunit[ ]{_Ws}{_W*_s}
\printunit[ ]{_Wh}{_W*_h}

\printunit{_VA}{_V*_A}

\printunit{_As}{_A*_s}
\printunit[ ]{_Ah}{_A*_h}

\printunit[Information]{_nibble}{4*_bit}
\printunit[ ]{_B}{8*_bit}

\printunit[Information \newline Transfer Rate]{_bps}{_bit/_s}

\printunit[Number]{_percent}{1e-2*_1}
\printunit[ ]{_permille}{1e-3*_1}
\printunit[ ]{_ppm}{1e-6*_1}
\printunit[ ]{_ppb}{1e-9*_1}
\printunit[ ]{_ppt}{1e-12*_1}
\printunit[ ]{_ppq}{1e-15*_1}

\printunit[ ]{_dB}{_1}

\hline

\end{tabularx}
\end{table}










\newpage
\subsection{Imperial Units}
\label{ch:Imperial Units}

\begin{table}[H]
\centering
\begin{tabularx}{\linewidth}{%
  >{\setlength\hsize{1\hsize}}X%
  l%
  l%
  >{\setlength\hsize{1\hsize}}X%
}

\thead{Quantity} & \thead{Unit} & \thead{Symbol} & \thead{Definition} \\\hline

\printunit[Length]{_in}{0.0254*_m}
\printunit[ ]{_th}{0.001*_in}
\printunit[ ]{_pica}{_in/6}
\printunit[ ]{_pt}{_in/72}
\printunit[ ]{_hh}{4*_in}
\printunit[ ]{_ft}{12*_in}
\printunit[ ]{_yd}{3*_ft}
\printunit[ ]{_rd}{5.5*_yd}
\printunit[ ]{_ch}{4*_rd}
\printunit[ ]{_fur}{10*_ch}
\printunit[ ]{_mi}{8*_fur}
\printunit[ ]{_lea}{3*_mi}
\printunit[ ]{_nmi}{1852 * _m}
\printunit[ ]{_nlea}{3*_nmi}
\printunit[ ]{_cbl}{_nmi/10}
\printunit[ ]{_ftm}{6*_ft}

\printunit{_kn}{_nmi/_h}

\printunit{_ac}{43560*_ft^2}

\printunit{_gal}{4.54609*_L}
\printunit[ ]{_qt}{_gal/4}
\printunit[ ]{_pint}{_qt/2}
\printunit[ ]{_cup}{_pint/2}
\printunit[ ]{_gi}{_pint/4}
\printunit[ ]{_fl_oz}{_gi/5}
\printunit[ ]{_fl_dr}{_fl_oz/8}


\hline

\end{tabularx}
\end{table}


\begin{table}[H]
\centering
\begin{tabularx}{\linewidth}{%
  >{\setlength\hsize{1\hsize}}X%
  l%
  l%
  >{\setlength\hsize{1\hsize}}X%
}

\thead{Quantity} & \thead{Unit} & \thead{Symbol} & \thead{Definition} \\\hline

\printunit[Mass]{_gr}{64.79891*_mg}
\printunit[ ]{_lb}{7000*_gr}
\printunit[ ]{_oz}{_lb/16}
\printunit[ ]{_dr}{_lb/256}
\printunit[ ]{_st}{14*_lb}
\printunit[ ]{_qtr}{2*_st}
\printunit[ ]{_cwt}{4*_qtr}
\printunit[ ]{_ton}{20*_cwt}


\hline

\end{tabularx}
\caption{Imperial units}
\end{table}




\newpage
\subsection{U.S. customary units}
\label{ch:U.S. customary units}

In the U.S., the length units are bound  to the meter differently than in the imperial system. The followin definitions are taken from \url{https://en.wikipedia.org/wiki/United_States_customary_units}.


\begin{table}[H]
\centering
\begin{tabularx}{\linewidth}{%
  >{\setlength\hsize{1\hsize}}X%
  l%
  l%
  >{\setlength\hsize{1\hsize}}X%
}

\thead{Quantity} & \thead{Unit} & \thead{Symbol} & \thead{Definition} \\\hline

\printunit[Length]{_in_US}{_m/39.37}
\printunit[ ]{_hh_US}{4*_in_US}
\printunit[ ]{_ft_US}{3*_hh_US}
\printunit[ ]{_li_US}{0.66*_ft_US}
\printunit[ ]{_yd_US}{3*_ft_US}
\printunit[ ]{_rd_US}{5.5*_yd_US}
\printunit[ ]{_ch_US}{4*_rd_US}
\printunit[ ]{_fur_US}{10*_ch_US}
\printunit[ ]{_mi_US}{8*_fur_US}
\printunit[ ]{_lea_US}{3*_mi_US}
\printunit[ ]{_ftm_US}{72*_in_US}
\printunit[ ]{_cbl_US}{120*_ftm_US}

\printunit{_ac_US}{_ch_US*_fur_US}

\printunit{_gal_US}{231*_in^3}
\printunit[ ]{_qt_US}{_gal_US/4}
\printunit[ ]{_pint_US}{_qt_US/2}
\printunit[ ]{_cup_US}{_pint_US/2}
\printunit[ ]{_gi_US}{_pint_US/4}
\printunit[ ]{_fl_oz_US}{_gi_US/4}
\printunit[ ]{_Tbsp_US}{_fl_oz_US/2}
\printunit[ ]{_tsp_US}{_Tbsp_US/3}
\printunit[ ]{_fl_dr_US}{_fl_oz_US/8}


\hline

\end{tabularx}
\end{table}


\begin{table}[H]
\centering
\begin{tabularx}{\linewidth}{%
  >{\setlength\hsize{1\hsize}}X%
  l%
  l%
  >{\setlength\hsize{1\hsize}}X%
}

\thead{Quantity} & \thead{Unit} & \thead{Symbol} & \thead{Definition} \\\hline

\printunit[Mass]{_qtr_US}{25*_lb}
\printunit[Mass]{_cwt_US}{4*_qtr_US}
\printunit[Mass]{_ton_US}{20*_cwt_US}


\hline

\end{tabularx}
\caption{U.S. customary units}
\end{table}









\newpage
\subsection{International Currencies}
\label{ch:currencies}

\newcommand{\curfactor}[1]{%
  \texttt{\directlua{tex.print( (#1):to(_EUR):__tonumber() )}}%
}


\begin{table}[H]
\centering
\begin{tabularx}{\linewidth}{%
  l%
  l%
  l%
  l%
  >{\setlength\hsize{1\hsize}}X%
}

\thead{Quantity} & \thead{Unit} & \thead{Symbol} & \thead{Dimension} & \thead{Definition} \\\hline


currency &
Afghan afghani&
|_AFN| & 
$\C$ & 
\curfactor{_AFN}|*_EUR| \\

 &
Albanian lek &
|_ALL| & 
$\C$ & 
\curfactor{_ALL}|*_EUR| \\

 &
Armenian Dram &
|_AMD| & 
$\C$ & 
\curfactor{_AMD}|*_EUR| \\

 &
Angolan Kwanza &
|_AOA| & 
$\C$ & 
\curfactor{_AOA}|*_EUR| \\

 &
Argentine Peso &
|_ARS| & 
$\C$ & 
\curfactor{_ARS}|*_EUR| \\

 &
Australian dollar &
|_AUD| & 
$\C$ & 
\curfactor{_AUD}|*_EUR| \\

 &
Azerbaijani Manat &
|_AZN| & 
$\C$ & 
\curfactor{_AZN}|*_EUR| \\

 &
Bosnian Mark &
|_BAM| & 
$\C$ & 
\curfactor{_BAM}|*_EUR| \\

 &
Bangladeshi Taka &
|_BDT| & 
$\C$ & 
\curfactor{_BDT}|*_EUR| \\

 &
Burundian Franc &
|_BIF| & 
$\C$ & 
\curfactor{_BIF}|*_EUR| \\

 &
Bolivian Boliviano &
|_BOB| & 
$\C$ & 
\curfactor{_BOB}|*_EUR| \\

 &
Brazilian Real &
|_BRL| & 
$\C$ & 
\curfactor{_BRL}|*_EUR| \\

 &
Botswana Pula &
|_BWP| & 
$\C$ & 
\curfactor{_BWP}|*_EUR| \\

 &
Belarusian Ruble &
|_BYN| & 
$\C$ & 
\curfactor{_BYN}|*_EUR| \\

 &
Canadian Dollar &
|_CAD| & 
$\C$ & 
\curfactor{_CAD}|*_EUR| \\

 &
Congolese Franc &
|_CDF| & 
$\C$ & 
\curfactor{_CDF}|*_EUR| \\


 &
U.S. dollar &
|_USD| & 
$\C$ & 
|0.89*_EUR| \\

 &
Japanese yen &
|_JPY| & 
$\C$ & 
|0.008*_EUR| \\

 &
British pound &
|_GBP| & 
$\C$ & 
|1.17*_EUR| \\



 &
Canadian dollar &
|_CAD| & 
$\C$ & 
|0.66*_EUR| \\

 &
Swiss franc &
|_CHF| & 
$\C$ & 
|0.88*_EUR| \\

 &
Chinese yuan &
|_CNY| & 
$\C$ & 
|0.13*_EUR| \\

 &
Swedish krona &
|_SEK| & 
$\C$ & 
|0.094*_EUR| \\

 &
New Zealand dollar &
|_NZD| & 
$\C$ & 
|0.60*_EUR| \\




\hline

\end{tabularx}
\caption{Currency units based on exchange rates from 7.3.2019, 21:00 UTC.}
\end{table}


















% shortcut for method definitions
\newcommand{\method}[2]{\subsection*{|#1.#2|}}

\newcommand{\subtitle}[1]{\noindent \\\textbf{#1}}

% set listings language to lua
\lstset{language=Lua}




\newpage
\section{Lua Documentation}

In this chapter, the following shortcuts will be used.
\begin{lstlisting}
local D = physical.Dimension
local U = physical.Unit
local N = physical.Number
local Q = physical.Quantity
\end{lstlisting}

The term |number| refers to a lua integer or a lua float number. By |string| a lua string is meant and by |bool| a lua boolean.






\subsection{physical.Quantity}
The quantity class is the main part of the library. Each physical Quantity and all units are represented by an instance of this class.


\method{Q}{new(q=nil)}
\begin{quote}
  Copy Constuctor

  \subtitle{Parameters}
  \begin{description}
    \item |q| : |Q| or \ |number|, optional\\
      Optional argument is either |Q|, a |number| or |nil|.

    \item |return| : |Q|\\
      The created |Q| instance
  \end{description}

  \subtitle{Note}\\
  As an argument it takes |Q|, |number| or |nil|. If |Q| is given, a copy of it is made and returned. If a |number| is given, the function creates a dimeensionless quantity with that value. In the case |nil| is given, the quantity |_1| is returned.

  \subtitle{Example}
  \begin{lstlisting}
  myOne = Q()
  myNumber = Q(42)
  myLength = Q(73*_m)
  \end{lstlisting}
\end{quote}




\method{Q}{defineBase(symbol,name,dimension)}
\begin{quote}
  This function is used to declare the base units. Units are represented as |Q| instances.  

  \subtitle{Parameters}
  \begin{description}
  \item |symbol| : |string|\\
    symbol of the base quantity

  \item |name| : |string|\\
    name of the base quantity

  \item |dimension| : |D|\\
    Instance of the |D| class, which represents the dimension of the quantity.

  \item |return| : |Q|\\
    The created |Q| instance.
  \end{description}

  \subtitle{Note}\\
  The function creates a global variable, an underscore concatenated with the |symbol| argument, e. g. |m| becomes the global variable |_m|.

  The |name| is used for example in the siunitx conversion function, e.g |meter| will be converted to |\meter|. 

  Each quantity has a dimension associated with it. The argument |dimension| allows any dimension to be associated to base quantities. By default, the SI convention is used. 

  \subtitle{Example}
  \begin{lstlisting}
Q.defineBase("m", "meter", L)
Q.defineBase("kg", "kilogram", M)
  \end{lstlisting}
\end{quote}




\method{Quantity}{define(symbol, name, q, tobase=nil, frombase=nil)}
\begin{quote}
  Creates a new derived unit from an expression of other units. For affine quantities like the temperature in celcius, one can give convertion functions to and from base units.

  \subtitle{Parameters}
  \begin{description}
  \item |symbol| : |string|\\
  Symbol of the base quantity

  \item |name| : |string|\\
    Name of the base quantity

  \item |q| : |physical.Quantity|\\
    Definition of the unit

  \item |tobase| : |function|, optional\\
    to convert a quantity to base units

  \item |frombase| : |function|, optional\\
    to convert a quantity from the base units

  \item |return| : |Quantity|\\
    The defined quantity
  \end{description}

  \subtitle{Examples}
  \begin{lstlisting}
Q.define("L", "liter", _dm^3)
Q.define("Pa", "pascal", _N/_m^2)
Q.define("C", "coulomb", _A*_s)

Q.define(
  "degC", 
  "celsius",
  _K, 
  function(q)
    q.value = q.value + 273.15
    return q
  end,
  function(q)
    q.value = q.value - 273.15
    return q
  end
)
  \end{lstlisting}
\end{quote}






\method{Quantity}{definePrefix(symbol,name,factor)}
\begin{quote}
  Defines a new prefix.

  \begin{description}
  \item |symbol| : |string|, Symbol of the base quantity

  \item |name| : |string|, Name of the base quantity

  \item |factor| : |number|, the factor which corresponds to the prefix
  \end{description}


\begin{lstlisting}
Q.definePrefix("c", "centi", 1e-2)
Q.definePrefix("a", "atto", 1e-18)
\end{lstlisting}
\end{quote}





\method{Quantity}{addPrefix(prefixes, units)}
\begin{quote}
  Create several units with prefixes from a given unit.

  \begin{description}
  \item |prefixes| : |string|, list of unit symbols

  \item |units| : |Quantity|, list of quantities
  \end{description}


\begin{lstlisting}
Q.addPrefix({"n","u","m","k","M","G"},{_m,_s,_A})
\end{lstlisting}
\end{quote}



\method{Quantity}{to(self,q,usefunction=false)}
\begin{quote}
  Converts the quantity self to the unit of the quantity |q|. If the boolean |usefunction| is true, the convertion function is used for conversion.

  \begin{description}
  \item |self| : |Quantity| 
  \item |q| : |Quantity|
  \item |usefunction| : |Bool| 
  \end{description}

\begin{lstlisting}
s = 1.9 * _km
print( s:to(_m) )
`
\begin{luacode}
s = 1.9 * _km
tex.write(tostring(s:to(_m)) )
\end{luacode}
`

T = 10 * _degC
print( T:to(_K) )
`
\begin{luacode}
T = 10 * _degC
tex.write(tostring(T:to(_K)) )
\end{luacode}
`
print( T:to(_K,true) )
`
\begin{luacode}
T = 10 * _degC
tex.write(tostring(T:to(_K,true)) )
\end{luacode}
`
\end{lstlisting}

\end{quote}



\method{Quantity}{tosiunitx(self,param,mode)}
\begin{quote}
  Converts the quantity into a siunitx string.

  \begin{description}
  \item |self| : |Quantity| 
  \item |param| : |string|
  \item |mode| : |Number|, 0:\textbackslash SI, 1:\textbackslash num, 2:\textbackslash si
  \end{description}

\begin{lstlisting}
s = 1.9 * _km

print( s:tosiunitx() )
`
\begin{luacode}
s = 1.9 * _km
tex.write(tostring(s:tosiunitx()) )
\end{luacode}
`

print( s:tosiunitx(nil,1) )
`
\begin{luacode}
tex.write(tostring(s:tosiunitx(nil,1)) )
\end{luacode}
`

print( s:tosiunitx(nil,2) )
`
\begin{luacode}
tex.write(tostring(s:tosiunitx(nil,2)) )
\end{luacode}
`
\end{lstlisting}

\end{quote}






\method{Quantity}{isclose(self,q,r)}
\begin{quote}
  Checks if this quantity is close to another one. The argument |r| is the maximal relative deviation.

  \begin{description}
  \item |self| : |Quantity| 
  \item |q| : |Quantity,Number| 
  \item |r| : |Number| 
  \end{description}

\begin{lstlisting}
s_1 = 1.9 * _m
s_2 = 2.0 * _m
print( s_1:isclose(s_2,0.1) )
`
\begin{luacode}
s_1 = 1.9 * _m
s_2 = 2.0 * _m
tex.write(tostring(s_1:isclose(s_2,10 * _percent)) )
\end{luacode}
`
print( s_1:isclose(s_2,0.01) )
`
\begin{luacode}
s_1 = 1.9 * _m
s_2 = 2.0 * _m
tex.write(tostring(s_1:isclose(s_2,1 * _percent)) )
\end{luacode}
`
\end{lstlisting}

\end{quote}



\method{Quantity}{min(q1, q2, ...)}
\begin{quote}
  Returns the smallest quantity of several given ones. The function returns |q1| if the Quantities are equal.

  \begin{description}
  \item |q1| : |Quantity,Number|, first argument

  \item |q2| : |Quantity,Number|, second argument
  \end{description}


\begin{lstlisting}
s_1 = 15 * _m
s_2 = 5 * _m
print(s_1:min(s_2))
`
\begin{luacode}
s_1 = 15 * _m
s_2 = 5 * _m
tex.write(tostring(s_1:min(s_2)))
\end{luacode}
`
\end{lstlisting}
\end{quote}


\method{Quantity}{max(q1, q2, ...)}
\begin{quote}
  Returns the biggest quantity of several given ones. The function returns |q1| if the Quantities are equal.

  \begin{description}
  \item |q1| : |Quantity,Number|, first argument

  \item |q2| : |Quantity,Number|, second argument
  \end{description}


\begin{lstlisting}
s_1 = 15 * _m
s_2 = 5 * _m
print(s_1:max(s_2))
`
\begin{luacode}
s_1 = 15 * _m
s_2 = 5 * _m
tex.write(tostring(s_1:max(s_2)))
\end{luacode}
`
\end{lstlisting}
\end{quote}




\method{Quantity}{abs(q)}
\begin{quote}
  Returns the absolute value of the given quantity |q|.

  \begin{description}
  \item |q| : |Quantity,Number|, argument
  \end{description}


\begin{lstlisting}
U = -5 * _V
print(U)
`
\begin{luacode}
U = -5 * _V
tex.write(tostring(U))
\end{luacode}
`
print(U:abs())
`
\begin{luacode}
U = -5 * _V
tex.write(tostring(U:abs()))
\end{luacode}
`
\end{lstlisting}
\end{quote}




\method{Quantity}{sqrt(q)}
\begin{quote}
  Returns the square root of the given quantity.

  \begin{description}
  \item |q| : |Quantity,Number| argument
  \end{description}


\begin{lstlisting}
A = 25 * _m^2
s = sqrt(A)
print(s)
`
\begin{luacode}
A = 25 * _m^2
s = sqrt(A)
tex.write(tostring(s))
\end{luacode}
`
\end{lstlisting}
\end{quote}



\method{Quantity}{log(q, base)}
\begin{quote}
  Returns the logarithm of the given quantitiy. If no base is given, the natural logarithm is calculated.

  \begin{description}
  \item |q| : |Quantity,Number| dimensionless argument
  \item |base| : |Quantity,Number| dimensionless argument
  \end{description}


\begin{lstlisting}
I = 1 * _W/_m^2
I_0 = 1e-12 * _W/_m^2
print(10 * (I/I_0):log(10) * _dB )
`
\begin{luacode}
I = 1 * _W/_m^2
I_0 = 1e-12 * _W/_m^2
tex.write(tostring(10 * (I/I_0):log(10.0) *_dB ))
\end{luacode}
`
\end{lstlisting}
\end{quote}




\method{Quantity}{exp(q)}
\begin{quote}
  Returns the value of the exponential function of the given quantitiy.

  \begin{description}
  \item |q| : |Quantity,Number| dimensionless argument
  \end{description}

\begin{lstlisting}
x = 2 * _1
print( x:exp() )
`
\begin{luacode}
x = 2 * _1
tex.write(tostring(x:exp()))
\end{luacode}
`
\end{lstlisting}

\end{quote}



\method{Quantity}{sin(q)}
\begin{quote}
  Returns the value of the sinus function of the given quantitiy.

  \begin{description}
  \item |q| : |Quantity,Number| dimensionless argument
  \end{description}

\begin{lstlisting}
alpha = 30 * _deg
print( alpha:sin() )
`
\begin{luacode}
alpha = 30 * _deg
tex.write(tostring(alpha:sin()))
\end{luacode}
`
\end{lstlisting}

\end{quote}



\method{Quantity}{cos(q)}
\begin{quote}
  Returns the value of the cosinus function of the given quantity. The quantity has to be dimensionless.

  \begin{description}
  \item |q| : |Quantity,Number| dimensionless argument
  \end{description}

\begin{lstlisting}
alpha = 60 * _deg
print( alpha:cos() )
`
\begin{luacode}
alpha = 60 * _deg
tex.write(tostring(alpha:cos()))
\end{luacode}
`
\end{lstlisting}

\end{quote}



\method{Quantity}{tan(q)}
\begin{quote}
  Returns the value of the tangent function of the given quantity. The quantity has to be dimensionless.

  \begin{description}
  \item |q| : |Quantity,Number| dimensionless argument
  \end{description}

\begin{lstlisting}
alpha = 45 * _deg
print( alpha:tan() )
`
\begin{luacode}
alpha = 45 * _deg
tex.write(tostring(alpha:tan()))
\end{luacode}
`
\end{lstlisting}

\end{quote}



\method{Quantity}{asin(q)}
\begin{quote}
  Returns the value of the arcus sinus function of the given quantity. The quantity has to be dimensionless.

  \begin{description}
  \item |q| : |Quantity,Number| dimensionless argument
  \end{description}

\begin{lstlisting}
x = 0.5 * _1
print( x:asin():to(_deg) )
`
\begin{luacode}
x = 0.5 * _1
tex.write(tostring(x:asin():to(_deg)))
\end{luacode}
`
\end{lstlisting}

\end{quote}



\method{Quantity}{acos(q)}
\begin{quote}
  Returns the value of the arcus cosinus function of the given quantity. The quantity has to be dimensionless.

  \begin{description}
  \item |q| : |Quantity,Number| dimensionless argument
  \end{description}

\begin{lstlisting}
x = 0.5 * _1
print( x:acos():to(_deg) )
`
\begin{luacode}
x = 0.5 * _1
tex.write(tostring(x:acos():to(_deg)))
\end{luacode}
`
\end{lstlisting}

\end{quote}




\method{Quantity}{atan(q)}
\begin{quote}
  Returns the value of the arcus tangent function of the given quantity. The quantity has to be dimensionless.

  \begin{description}
  \item |q| : |Quantity,Number| dimensionless argument
  \end{description}

\begin{lstlisting}
x = 1 * _1
print( x:atan():to(_deg) )
`
\begin{luacode}
x = 1 * _1
tex.write(tostring(x:atan():to(_deg)))
\end{luacode}
`
\end{lstlisting}

\end{quote}




\method{Quantity}{sinh(q)}
\begin{quote}
  Returns the value of the hyperbolic sine function of the given quantity. The quantity has to be dimensionless. Since lua doesn't implement the hyperbolic functions the following formula is used 
  $$
    \sinh(x) = 0.5 \cdot e^x - 0.5 / e^x  \quad.
  $$

  \begin{description}
  \item |q| : |Quantity,Number| dimensionless argument
  \end{description}

\begin{lstlisting}
x = 1 * _1
print( x:sinh() )
`
\begin{luacode}
x = 1 * _1
tex.write(tostring(x:sinh()))
\end{luacode}
`
\end{lstlisting}

\end{quote}



\method{Quantity}{cosh(q)}
\begin{quote}
  Returns the value of the hyperbolic cosine function of the given quantity. The quantity has to be dimensionless. Since lua doesn't implement the hyperbolic functions the following formula is used 
  $$
    \cosh(x) = 0.5 \cdot e^x + 0.5 / e^x  \quad.
  $$

  \begin{description}
  \item |q| : |Quantity,Number| dimensionless argument
  \end{description}

\begin{lstlisting}
x = 1 * _1
print( x:cosh() )
`
\begin{luacode}
x = 1 * _1
tex.write(tostring(x:cosh()))
\end{luacode}
`
\end{lstlisting}

\end{quote}



\method{Quantity}{tanh(q)}
\begin{quote}
  Returns the value of the hyperbolic tangent function of the given quantity. The quantity has to be dimensionless. Since lua doesn't implement the hyperbolic functions the following formula is used 
  $$
    \tanh(x) = \frac{e^x - e^{-x}}{e^x + e^{-x}} \quad.
  $$


  \begin{description}
  \item |q| : |Quantity,Number| dimensionless argument
  \end{description}

\begin{lstlisting}
x = 1 * _1
print( x:tanh() )
`
\begin{luacode}
x = 1 * _1
tex.write(tostring(x:tanh()))
\end{luacode}
`
\end{lstlisting}

\end{quote}




\method{Quantity}{asinh(q)}
\begin{quote}
  Returns the value of the inverse hyperbolic sine function of the given quantity. The quantity has to be dimensionless. Since lua doesn't implement the hyperbolic functions the following formula is used 
  $$
    \text{asinh}(x) = \ln\left( x + \sqrt{x^2 + 1} \right)  \quad.
  $$


  \begin{description}
  \item |q| : |Quantity,Number| dimensionless argument
  \end{description}

\begin{lstlisting}
x = 1 * _1
print( x:asinh() )
`
\begin{luacode}
x = 1 * _1
tex.write(tostring(x:asinh()))
\end{luacode}
`
\end{lstlisting}

\end{quote}



\method{Quantity}{acosh(q)}
\begin{quote}
  Returns the value of the inverse hyperbolic cosine function of the given quantity. The quantity has to be dimensionless. Since lua doesn't implement the hyperbolic functions the following formula is used 
  $$
    \text{acosh}(x) = \ln\left( x + \sqrt{x^2 - 1} \right)  \quad, x > 1 \quad.
  $$


  \begin{description}
  \item |q| : |Quantity,Number| dimensionless argument bigger than or equal to one.
  \end{description}

\begin{lstlisting}
x = 2 * _1
print( x:acosh() )
`
\begin{luacode}
x = 2 * _1
tex.write(tostring(x:acosh()))
\end{luacode}
`
\end{lstlisting}

\end{quote}



\method{Quantity}{atanh(q)}
\begin{quote}
  Returns the value of the inverse hyperbolic cosine function of the given quantity. The quantity has to be dimensionless. Since lua doesn't implement the hyperbolic functions the following formula is used 
  $$
    \text{atanh}(x) = \ln\left( \frac{1 + x}{1 - x} \right)  \quad, -1 < x < 1 \quad.
  $$

  \begin{description}
  \item |q| : |Quantity,Number| dimensionless argument with magnitude smaller than one.
  \end{description}

\begin{lstlisting}
x = 0.5 * _1
print( x:atanh() )
`
\begin{luacode}
x = 0.5 * _1
tex.write(tostring(x:atanh()))
\end{luacode}
`
\end{lstlisting}

\end{quote}



%
%
%
%
\subsection{physical.Dimension}

 All physical quantities do have a physical dimension. For example the quantity \emph{Area} has the dimension $L^2$ (lenght to the power of two). In the SI-System there are seven base dimensions, from which all other dimensions are derived. Each dimension is represented by an $n$-tuple, where $n$ is the number of base dimensions. Each physical quantity has an associated dimension object. It is used two check if two quantities can be added or subtraced and if they are equal. 


\method{Dimension}{new(q=nil)}
\begin{quote}
  Constructor of the |Dimension| class.

  \subtitle{Parameters}
  \begin{description}
  \item |q| : |Dimension| or |string|, optional\\
    The name or symbol of the dimension. If |q| is a dimension, a copy of it is made. If no argument ist given, a dimension \emph{zero} is created.

  \item |return| : |Dimension|\\
    The created |Quantity| object
  \end{description}

  \subtitle{Notes}\\
  --

  \subtitle{Examples}
  \begin{lstlisting}
  V_1 = D("Velocity")
  L = D("L")
  V_2 = D(L/T)
  \end{lstlisting}
\end{quote}








%
% physical.Unit
%
\subsection{physical.Unit}

The task of this class is keeping track of the unit term. The unit term is a fraction of units. The units in the enumerator and denominator can have an exponent. 


\method{Unit}{new(u=nil)}
\begin{quote}
  Copy Constructor. It copies a given unit object. If nothing is given, an empty unit is created.

  \subtitle{Parameters}
  \begin{description}
  \item |u| : |Unit|\\
    The unit object which will be copied.

  \item |return| : |Unit|\\
    The created |Unit| object
  \end{description}

\end{quote}


\method{Unit}{new(symbol, name, prefixsymbol=nil, prefixname=nil)}
\begin{quote}
  Constructor. A new |Unit| object with symbol is created. The prefixsymbol and prefixname are optional. 

  \subtitle{Parameters}
  \begin{description}
  \item |symbol| : |String|\\
    The symbol of the unit.

  \item |name| : |String|\\
    The name of the unit.

  \item |prefixsymbol| : |String|\\
    The optional symbol of the prefix. 

  \item |prefixname| : |String|\\
    The optional name of the prefix. 

  \item |return| : |Unit|\\
    The created |Unit| object
  \end{description}

\end{quote}




\method{Unit}{tosiunitx(self)}
\begin{quote}
  The unit term will be compiled into a string, which the LaTeX package siunitx can understand.

  \subtitle{Parameters}
  \begin{description}
  \item |return| : |String|\\
    The siunitx representation of the unit term.
  \end{description}

\end{quote}




%
% physical.Number
%
\subsection{physical.Number}

It does arithmetics with gaussian error propagation. A number instance has a mean value called |x| and an uncertainty called |dx|.

\method{Number}{new(n=nil)}
\begin{quote}
  This is the copy Constructor. It copies a given number object. If |n| is |nil|, an instance representing number zero with uncertainty zero is created.

  \subtitle{Parameters}
  \begin{description}
  \item |n| : |Number|\\
    The number object to be copied.

  \item |return| : |Number|\\
    The created |Number| instance.
  \end{description}

\end{quote}


\method{Number}{new(x, dx)}
\begin{quote}
  This constructor, creates a new instance with mean value |x| and uncertainty |dx|.

  \subtitle{Parameters}
  \begin{description}
  \item |x| : |number|\\
    mean value

  \item |dx| : |number|\\
    uncertainty value

  \item |return| : |Number|\\
    The created |Number| instance.
  \end{description}

  \subtitle{Examples}
  \begin{lstlisting}
  n = N(12,0.1)
  print(n)
  \end{lstlisting}

\end{quote}

\method{Number}{new(str)}
\begin{quote}
  This constructor creates a new instance from a string. It can parse strings of the form |3.4|, |3.4e-3|, |5.4e-3 +/- 2.4e-6|, |5.45(7)e-23|. 

  \subtitle{Parameters / Return}
  \begin{description}
  \item |str| : |string|\\
    The number as a string.

  \item |return| : |Number|\\
    The created |Number| object
  \end{description}

  \subtitle{Examples}
  \begin{lstlisting}
  n_1 = N("12.3e-6")
  print(n_1)

  n_2 = N("12 +/- 0.1")
  print(n_2)

  n_3 = N("12.0(1)")
  print(n_3)
  \end{lstlisting}
\end{quote}



\method{Number}{mean(n)}
\begin{quote}
  Returns the mean value

  \subtitle{Parameters / Return}
  \begin{description}
  \item |return| : |number|\\
    The mean value
  \end{description}
\end{quote}

\method{Number}{uncertainty(n)}
\begin{quote}
  Returns the uncertainty value

  \subtitle{Parameters / Return}
  \begin{description}
  \item |return| : |number|\\
    The uncertainty value
  \end{description}
\end{quote}


\method{Number}{abs(n)}
\begin{quote}
  Returns the absolute value of the number.

  \subtitle{Parameters / Return}
  \begin{description}
  \item |return| : |number|\\
    The absolute value
  \end{description}
\end{quote}

\method{Number}{sqrt(n)}
\begin{quote}
  Returns the square root of the number.

  \subtitle{Parameters / Return}
  \begin{description}
  \item |return| : |number|\\
    The square root
  \end{description}
\end{quote}





\bibliographystyle{plain}
\bibliography{lua-physical}





\end{document}

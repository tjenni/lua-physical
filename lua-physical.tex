%!TEX program = lualatex
\documentclass{ltxdoc}

\usepackage{url}
\usepackage[english]{babel}
\usepackage{hyperref}
\usepackage{luacode}
\usepackage{framed}
\usepackage{tcolorbox}
\usepackage{siunitx}


% siunitx config
\sisetup{
	output-decimal-marker = {.}, 
	per-mode = symbol,
	separate-uncertainty = false,
	add-decimal-zero = true,
	exponent-product = \cdot,
	round-mode=off
}

% use listings
\tcbuselibrary{listings,breakable}

\lstset{
  numberstyle=\footnotesize\color{vert},
  keywordstyle=\ttfamily\bfseries\color{blue},
  basicstyle=\ttfamily\footnotesize,
  commentstyle=\itshape\color{vert},
  stringstyle=\ttfamily,
  tabsize=2,
  showstringspaces=false,
  language=[LaTeX]TeX,
  breaklines=true,
  breakindent=30pt,
  defaultdialect=[LaTeX]TeX,
  morekeywords={}
}

\lstdefinelanguage{lua}
{morekeywords={for,end,function,do,if,else,elseif,then,
    tex.print,tex.sprint,io.read,io.open,string.find,string.explode,require},
  morecomment=[l]{--},
  morecomment=[s]{--[[}{]]},
  morestring=[b]''
}


\newtcblisting{latexcode}{
  arc=0pt,outer arc=0pt,
  colback=red!2!white,
  colframe=red!75!black,
  breakable,
  boxsep=0pt,left=5pt,right=5pt,top=5pt,bottom=5pt, bottomtitle =
  3pt, toptitle=3pt,
  boxrule=0pt,bottomrule=0.5pt,toprule=0.5pt, toprule at break =
  0pt, bottomrule at break = 0pt,
  listing only,boxsep=0pt,listing
  options={breaklines}
}

\begin{luacode*}
physical = require("physical")
\end{luacode*}

\newcommand{\q}[1]{%
	\directlua{tex.print(physical.Quantity.tosiunitx(#1,"add-decimal-zero=true,scientific-notation=fixed,exponent-to-prefix=false"))}%
}




\begin{document}
	\title{The \textsc{lua-physical} library \\\ \\\normalsize Version 0.1}
	\author{Thomas Jenni}
	\date{\today}
	\maketitle



\begin{abstract}
|lua-physical| is a pure Lua library which provides functions and object for doing computation with physical quantities. This package provides a lot of units of the SI-system and the imperial system. 
\end{abstract}

\tableofcontents






\newpage
\section{Introduction}

A physical quantitiy is composed by a number and a unit. This lua-package provides the means to do calculations. 

\begin{latexcode}
\begin{luacode}
	s = 10 * _m
	t = 2 * _s
	v = s/t
\end{luacode}

A car travels $\Q{s}$ in $\Q{t}$. Calculate its velocity.

$$
	v = \frac{s}{t} = \frac{\q{s}}{\q{t}} = \q{v}
$$
\end{latexcode}

\begin{luacode*}
s = 10 * _m
t = 2 * _s
v = s/t
\end{luacode*}



A car travels $\q{s}$ in $\q{t}$. Calculate its velocity.

$$
v = \frac{s}{t} 
= \frac{\q{s}}{\q{t}}
= \q{v}
$$

\newpage

\end{document}
